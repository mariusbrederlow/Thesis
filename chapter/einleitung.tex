\newpage
\section{Einleitung}
\subsection{Motivation und Zielsetzung}
Wenn wir heutzutage über die Verwendung und Vorteile von Hypermedia sprechen ist dies keine Innovation der letzten Jahre. Bereits im Jahr 1990 beschrieb Tim Berners-Lee in seinem Vorschlag für ein Hypertext-Projekt die wesentlichen Eigenschaften einer Hypertext basierten Anwendung:\\

"The current incompatibilities of the platforms and tools make it impossible to access existing information through a common interface, leading to waste of time, frustration and obsolete answers to simple data lookup. There is a potential large benefit from the integration of a variety of systems in a way which allows a user to follow links pointing from one piece of information to another one. This forming of a web of information nodes rather than a hierarchical tree or an ordered list is the basic concept behind HyperText."\footnote{Berners-Lee, T. (1990), \url{http://www.w3.org/Proposal.html}}\\

Die Idee von Berners-Lee ist eine Anwendung nicht in Form von sequentiellen Abläufen und einer hierarchischen Struktur abzubilden, sondern als eine Art Sammlung von lose gekoppelten Ressourcen mit der Möglichkeit sich in dieser Sammlung vor und zurück zu bewegen. Der Vorschlag bewegt sich damit weg von den hierarchischen Strukturen einer verketteten Liste oder der Darstellung als Baum. Diese Idee legte seinerzeit den Grundstein für das, was wir heute als das WorldWideWeb kennen. Das Internet ist wohl das größte und umfassendste, verteilte Informationssystem unserer Zeit. Der Erfolg des Internet beruht sicherlich auf der Tatsache, dass es möglich ist verschiedenste Arten Informationen rund um den Globus miteinander zu vernetzen und abrufbar zu machen.\\
Im Jahr 2000 hat sich Roy Fielding im Rahmen seiner Dissertation mit dem Thema beschäftigt, welche Architektur das Internet selbst besitzt und wie diese Architektur aussehen sollte. Das Internet ist für ihn per Definition ein verteiltes Hypermedia-System, wobei er Hypermedia als eine Weiterentwicklung von Hypertext betrachtet. Hypermedia umfasst im Gegensatz zu Hypertext nicht nur Verweise auf anderen Hypertext sondern auf Multimediale Inhalte wie Text, Audio und Video. Fielding selbst schlägt als Architektur einer verteilten Hypermedia Anwendung, das von ihm beschriebene "Representational State Transfer" (REST) vor. Ein zentraler Bestandteil dieser Architektur ist die Repräsentation von Ressourcen und die Steuerung der Anwendung durch den Einsatz von Hypermedia.\\
Dieser Architekturstil gewann in den letzten Jahren zunehmend an Bedeutung bei der Erstellung von verteilten Hypermedia Systemen. Mensch zu Maschine Kommunikation und reine Maschinenkommunikation kann gleichermaßen abgebildet werden. Eine Idee beim Einsatz von Hypermedia in verteilten Informationssystemen besteht darin das Verhalten beim Browsen durch das Internet auch außerhalb des Browsers anwendbar zu machen. Bei der Verwendung eines Browsers kann ein Mensch anhand von Beschreibungen oder dem Kontext der Information einem Verweis folgen. Das Folgen der Verweise ist nicht sequentiell sonder steht immer im Kontext der eingebetteten Informaltion. Mit anderen Worten, ein Mensch kann die Semantik der Information verstehen und dem gewünschten Link folgen. Dies ist möglich ohne den exakten Endpunkt des Verweises zu kennen. An dieser Stelle spiegelt sich auch ein wesentlicher Erfolgsfaktor des Internets bzw. der Nutzung von Hypertext wieder. Niemand kann sich eine große Anzahl von URIs (Uniform Ressource Identifier) merken. Selbst wenn jemand sich alle wichtigen URIs merken könnte ist nicht sichergestellt, dass sie über die Zeit stabil sind. Stabilität ist auch nicht notwendig, da das Folgen eines Links ohne den exakten Endpunkt zu kennen möglich ist.\\
Betrachtet man eine Client-Server-Architektur als einen oft eigesetzten Stil für verteilte Anwendungen, muss der Client seine verwendeten Endpunkte auf den Servern kennen. Bei vielen Client Applikationen sind diese Endpunkte als ein Bestandteil des Quellcodes festgelegt. Diese Tatsache hat zur Folge, dass im Fall einer Änderung oder Erweiterung des Adressraums die im Client festgelegten URIs invalidiert werden. Demnach besteht schon aufgrund der statischen Benutzung von URIs eine recht enge Kopplung zwischen Client und Server. Um einen lose gekoppelten Client zu entwerfen, sollte er keine festgelegten URIs aus dem Adress-Schema des Servers enthalten. Die einzige Ausnahme ist hier ein zentraler Eintrittspunkt, von dem alle weiterführenden Aktionen ausgehen. Die weiteren Operationsmöglichkeiten und die Navigation durch die Anwendung sollte vom Server dynamisch zur Laufzeit an den Client gesendet werden.\\
Dieses Vorgehen ist bei einer Mensch-zu-Maschine Kommunikation intuitiv umsetzbar. Es entspricht exakt der Anwendung eines Browsers um durch verschiedene Webseiten als Repräsentation von Informationen im Internet zu navigieren. Intuitive Benutzbarkeit kommt zu stande, weil ein Mensch in der Lage ist den Kontext der Information zu interpretieren und entsprechend zu handeln.\\
Bei einer Maschine-zu-Maschine Kommunikation ist das nicht der Fall. Eine Maschine kann in der Lage sein einem Verweis zu folgen, sie kennt jedoch nicht die Semantik des Verweises. Hierfür ist zusätzliches Wissen auf der Seite des Clients erforderlich. Bei vielen Anwendungen ist dieses Wissen in einer umfassenden Dokumentation festgelegt. Die Dokumentation beschreibt, welche Verweise und Endpunkte für eine bestimmte Anwendungslogik zur Verfügung stehen. Client Applikationen müssen dieses Wissen implementieren um die Nachrichten der Anwendung zu verstehen und die Anwendung zu steuern. Lose gekoppelte Systeme sind nach diesem Modell nur schwer zu entwerfen. Änderungen an den verwendeten Adressen müssen im Client stets reimplementiert werden.\\
Hypermedia setzt an dieser Stelle auf ein anderes Prinzip. Um die Implementation von Wissen auf der Clientseite möglichst gering zu halten werden selbstbeschreibende Hypermedianachrichten eingesetzt. Selbstbeschreibende Nachrichten sind angereichert mit semantischen Informationen für die Verarbeitung der Nachricht. Also mit dem Wissen das der Client benötigt um die erhaltenen Informationen darzustellen und die Anwendung zu steuern. Sämtliches, in einer externen Dokumentation festgelegtes Wissen, das nicht als ein Bestandteil der selbstbeschreibenden Nachricht mitgeliefert wird, ist im Hypermedia Umfeld als "Out of Band Knowledge" bezeichnet. Der Anteil dieser Wissensform sollte im Allgemeinen möglichst gering gehalten werden, um eine zu starke Kopplung zu vermeiden.\\
Reduziert man eine verteilte Hypermedia Anwendung auf das wesentliche Vorgehen, ergeben sich zwei charakteristische Eigenschaften. Auf der einen Seite ein Client, welcher in der Lage ist Informationen darzustellen und Verweisen zu folgen. Auf der anderen Seite ein Server, der Informationen und Verweise auf seine Ressourcen zur Verfügung stellt. Als ein zentraler Bestandteil der Kommunikation können somit Verweise identifiziert werden. Bestandteile eines Verweises selbst lassen sich anhand der genannten Eigenschaften einer verteilten Hypermedia Anwendung abstrakt ein drei Attribute unterteilen:
\begin{itemize}
\item Der URI als aktueller vom Server zur Verfügung gestellter Endpunkt für eine Ressource
\item Das Relationsattribut zur Beschreibung der Beziehung zwischen aktuell genutzter und Zielressource
\item Das Methoden Attribut um festzulegen mit welcher Protokollmethode die Ressource angesprochen werden kann
\end{itemize}
Verweise, angereichert mit der beschriebenen Semantik, lösen die Kopplung zwischen Client und Server bis zu einem gewissen Grad. Clients einer verteilten Hypermedia Anwendung benötigen im Idealfall nur Kenntnis über die verwendeten Relationsattribute, wobei unterstellt wird, dass der Client den zentralen Eintrittspunkt der Anwendung kennt und der Server seine Adressen komplett selbständig verwalten kann. Die Steuerung des Applikationsfluss erfolgt ausschließlich über die vom Server zur Verfügung gestellten Verweise. Dieses Vorgehen entspricht dem eines Browsers und einer Hypermedia getriebenen Anwendung gleichermaßen.\\
Hypermedia birgt viel Potential in Hinblick auf die Erstellung eines verteilten Informationssystems. Durch semantisch angereicherte Nachrichten wird versucht die Kopplung zwischen Client und Server auf ein Minimum zu reduzieren. Ein her geht damit auch eine bessere Wartbarkeit und Erweiterbarkeit der Applikation. Essentieller Grundbestandteil der Kommunikation ist ein Hypermediaformat, das möglichst viele der geforderten Eigenschaften unterstützt und einen lose gekoppelten Nachrichtenaustausch ermöglicht. Aktuell existieren viele verschiedene Formate, die den Erfordernissen in unterschiedlichen Umfängen genügen. Diese Arbeit hat zum Ziel eine Beurteilungsgrundlage für Hypermediaformate zu entwickeln, verschiedene Formate miteinander zu vergleichen und auf ihre Leistungsfähigkeit hin zu untersuchen.











