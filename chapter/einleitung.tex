\newpage
\section{Einleitung}
\subsection{Motivation und Zielsetzung}

Das Internet ist wohl das größte und umfassendste Informationssystem unserer Zeit. Der Erfolg des Internet beruht sicherlich auf der Tatsache, dass es möglich ist verschiedenste Informationen rund um den Globus miteinander zu vernetzen und abrufbar zu machen. Heutzutage kann sich niemand mehr ein Internet ohne RSS-Feed, Podcasts, Youtube und diverse andere Multimediale Inhalte vorstellen. Zudem ist das Internet per Definition ein hochgradig verteiltes System. Dieser Erfolg ist in erheblichem Umfang den verschiedensten Hypermedia-Formaten zuzuschreiben. Viele große und hochgradig skalierte Webanwendungen laufen auf der Basis von Hypermedia oder bieten zum Teil Hypermedia APIs als öffentliche Schnittstellen zu ihrem Dienst an. Einige bekannte Vertreter sind z.B. Amazon, Facebook und Twitter. Mit der Einführung von Representational State Transfer (REST) als ein Architekturansatz für verteilte Informationssysteme  im Jahre 2000 durch Roy Fielding, gewann das Thema Hypermedia in den letzten Jahren auch für kleinere Unternehmensanwendungen zunehmend an Bedeutung. Ein wesentlicher Punkt bei der Umsetzung von REST als Architekturstil ist Hypermedia. In der Dissertation von Fielding beschreibt er einen Ansatz, den er selbst "Hypermedia as the engine of application state" (HATEOAS) nennt. Grundlegend geht es hierbei darum, Hypermedia zu verwenden um Informationen auszutauschen und den Status der Anwendung zu ändern. Dabei sollte das Hypermediaformat sekbsterklärend und möglichst unabhängig vom Transportprotokoll sein. Von selbsterklärend im Kontext von Hypermedia spricht man, wenn das Hypermediaformat semantische und kontextbezogene Informationen mit sich bringt. \\
Für den Begriff Hypermedia selbst gibt es keine Norm oder Standardisierung, die beschreibt, was Hypermedia überhaupt umfassen muss. Daher tummeln sich am Markt verschiedenste Hypermediaformate mit unterschiedlichsten Charakteristiken und Möglichkeiten. Diese Arbeit hat zum Ziel eine Beurteilungsgrundlage für Hypermediaformate zu schaffen, verschiedene Formate miteinander zu vergleichen und auf ihre Leistungsfähigkeit hin zu untersuchen.


	\begin{itemize}
		\item Erläuterung des IST-Zustandes und
		\item der daraus resultierenden Probleme\tocite{HMF92}{11}	
	\end{itemize}
